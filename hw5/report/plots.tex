\subsection{Plots from task 1.2}
\newcommand{\picScalen}{0.36}

\begin{figure}[h!]
\label{fig:opg12_nomoto_0.43}
\center
\subfloat[r]{
\includegraphics[scale=\picScalen]{./figures/task12_nomoto2_r_step0.43.eps}
}
\subfloat[$\psi$]{
\includegraphics[scale=\picScalen]{./figures/task12_nomoto2_psi_step0.43.eps}
} \\
\subfloat[r]{
\includegraphics[scale=\picScalen]{./figures/task12_nomoto2_r_step0.1.eps}
}
\subfloat[$\psi$]{
\includegraphics[scale=\picScalen ]{./figures/task12_nomoto2_psi_step0.1.eps}
}\\
\subfloat[r]{
\includegraphics[scale=0.5]{./figures/task12_nomoto2_r_step0.05.eps}
}
\subfloat[$\psi$]{
\includegraphics[scale=\picScalen]{./figures/task12_nomoto2_psi_step0.05.eps}
}

\caption{Plot of curve fitting procedure in task 1.2 of the $H(s)$ model for different step response $\delta$}
\end{figure}
\newcommand{\picScale}{0.36}



\begin{figure}[h!]
\label{fig:opg12_nomotoc_0.43}
\center
\subfloat[r]{
\includegraphics[scale=\picScale ]{./figures/task12_nomoto2c_r_step0.43.eps}
}
\subfloat[$\psi$]{
\includegraphics[scale=\picScale]{./figures/task12_nomoto2c_psi_step0.43.eps}
} \\
\subfloat[r]{
\includegraphics[scale=\picScale]{./figures/task12_nomoto2c_r_step0.1.eps}
}
\subfloat[$\psi$]{
\includegraphics[scale=\picScale ]{./figures/task12_nomoto2c_psi_step0.1.eps}
}\\
\subfloat[r]{
\includegraphics[scale=\picScale]{./figures/task12_nomoto2c_r_step0.05.eps}
}
\subfloat[$\psi$]{
\includegraphics[scale=\picScale]{./figures/task12_nomoto2c_psi_step0.05.eps}
}

\caption{Plot of curve fitting procedure in task 1.2 of the $H_c(s)$ model for different step response $\delta$}
\end{figure}

 \clearpage


\subsection{Plots from task 1.4}
\begin{figure}[h!]
\center
\subfloat[$\psi$]{
\includegraphics[scale=0.65 ]{./figures/task14_psi.eps}
}\\
\subfloat[$r$]{
\includegraphics[scale=0.65]{./figures/task14_r.eps}
}
\caption{Plots from simulating the heading of the ship with a PID controller}
\end{figure}

\subsection{Plots from task 1.6}
\begin{figure}[h!]

\center
\subfloat[$n_c = 3$]{
\includegraphics[scale=\picScale ]{./figures/opg1_6_nc_3.eps}
}
\subfloat[$n_c = 5$]{
\includegraphics[scale=\picScale ]{./figures/opg1_6_nc_5.eps}
}\\
\subfloat[$n_c = 7$]{
\includegraphics[scale=\picScale ]{./figures/opg1_6_nc_7.eps}
}
\subfloat[$n_c = 9$]{
\includegraphics[scale=\picScale ]{./figures/opg1_6_nc_9.eps}
}\\
\subfloat[$n_c = 9.5$]{
\includegraphics[scale=\picScale ]{./figures/opg1_6_nc_9.5.eps}
}
\caption{Comparison between model and MS fart\o ystyring \label{fig:opg1_6}}
\label{fig:opg14}
\end{figure}

\newpage
\subsection{Plots from task 1.8}
\begin{figure}[h!]

\center
\subfloat[$\tilde{u}$]{
\includegraphics[scale=\picScale ]{./figures/opg1_8_u_tilde.eps}
}
\subfloat[$u\ and\ u_d$]{
\includegraphics[scale=\picScale ]{./figures/opg1_8_u_and_u_d.eps}
}
\caption{Closed-loop behaviour \label{fig:opg1_8}} 
\end{figure}

\newpage
\subsection{Plots from Task 2.7}
\begin{figure}[h!] 
	\center 
	\includegraphics[scale=0.53, angle=-90]{./figures/pt_guidance.eps}
	\caption{Structure of the Path tracking model in Simulink.} 
	\label{fig:path_track_simulink_model}
\end{figure}
